
Patterns

==============

Notes were made in the 3-ring notebook on these patterns:

	Stewards of Earth (7)
	One World Citizenship (15)
	
==============

Other notebook notes

	26 June 2007
	6 July 2007
	12 July 2007

==============

% These patterns are akin to generalized principles in science except that they deal with complex phenomena involving human behavior. Ideally, the patterns should have at least some of the properties of generalized principles. To be consistent, all the patterns must conform to these parameters:

	% They should be inter-complimentary, to use Fuller's term. In other words, they should be consistent and not self-contradictory. No pattern should interfere with another.
	
	% They should be general enough that the special-case solutions they inspire can evolve over time.


=================================================
 PART 1
=================================================

% A World That Works
% How the world should look

% The categories in the first part are a big part of my structural challenge. If I can get the categories right, it will likely make it clear where the holes are in the patterns.

	main goals
		% what do we mean by "World That Works" in human terms

Integration with nature

	environmental-goals
		% we're not only ones on this Earth, so ...
	natural-places
		% technically, these could be part of the environmental goals

Politial subdivisions 

	community
	% maybe this should be thought of as "world community". That would lead to other ideas of regional and local communities.

Global economic engine

	world-service-industries
	macroeconomics

Regional government

	public-service
	stable-state
	human-rights
	local-infrastructure
	social-services


ethics
belief-systems
children
design-science
intellectual-resources

equal-access
	% maybe this is a group rather than a pattern. It could include ideas like Education for Everyone and Available Energy.

civilization
public-sector (government)
private-enterprise
non-profit sector

tracking-success

----

We begin with that part of the language which defines what a world that works for all humanity might look like. These patterns are too broad to be designed and built all at once, but they provide a context for the work that needs to be done and can act as guides for how we approach our individual contributions. If we integrate these patterns into our thinking, our collective works will begin to create a world with these patterns in it.


(success-indicators) Before we can say that we have a world that works, we have to define what that means;

	1  Highest Standard of Living
	   % I'm wondering if this should be Freedom to Flourish.
	2  Everyone Lifted Up
	3  Success as the Norm
	4  Ongoing Success
	5  Regenerative Success

(environmental-indicators) work to protect and maintain the natural systems that we depend upon;

	6  Stewards of Earth
	7  Healthy Natural Services
	8  Stable Climate
	9  Diversity of Life

(natural-places) within these natural systems, encourage the formation and protection of natural places that experience limited human impact;

	10  Wild Places
	11  Space for Other Life

(community) allow people to self-organize according to physical, cultural and political realities into regions that are small enough for individuals to have a say in how they are run;

	12  One-World Citizens
	13  Independent Regions
?	**  Invested Local Citizen   % new pattern
		% The local people need to be involved in thier community or it will not work. This is one of the principles outlined at water.org.

*(world-service-industries) work to provide basic needs globally by building the industries and infrastructure required to take advantage of global scale efficiencies;

==============
see p.1 of black notebook: 13 Feb 2007
==============

	14  Livingry Industrial Complex
		% maybe rename this "Humanitarian Industrial Complex"
	15  World Service Industries
	16  Global Energy Network

(macroeconomics) build sustainable economic systems by basing them on renewable energy and concepts compatible with the reality of finite physical resources;

	17  Energy Accounting
	18  Global Energy Budget
	19  Wealth as Capacity
	20  Growth By Ephemeralization
	21  Accountable Corporation
	22  Temporary Ownership Economy

(public-service) within the independent regions, establish ways to make sure people have a say in how things are done;

	23  Government
	24  Free Press
	25  Transparent Power

(stable-state) ensure that each independent region has the elements of a stable state;

	26  Standing Military   % rename to Militia
		% armies require bases and equipment and are expensive to maintain.
		% Cholmers Johnson, Nemesis: The Last Days of the American Republic
		% discussed on Talk of the Nation on 27 Feb 2007
		% a standing army focuses power in the central government
		% it also creates an economic incentive -- if not need -- for conflict
	27  Police Force
	28  Judicial System

(human-rights)

	29  Basic Human Rights
	30  Individual Influence
	31  Education for Everyone
?	xx  Room for Nonconformists   % new pattern
		% We need for people who don't agree with these patterns to have a place in the society. Part of the role of civilization is to set limits on what behavior is acceptable. I certainly don't think anarchy is a good approach to creating a world that works. But we need to be very careful about what limit we choose.
		% In a sense, we are talking about a contract or a covenant. In exchange for a good life, we agree to play by the rules. Perhaps the way to do better is to make the rules explicit. Fuller did some of this with his discussion of how we have to prove our value by earning the right to exist, and how it is implicit that we will have to take from others so we can survive.

(local-infrastructure) build the infrastructure that is local and needs to be in local control;

	32  Sewage Treatment

(social-services) ensure that important social services are available regardless of where people live;

	33  Available Energy
	34  Available Health Care
		% should rename to "Accessible Health Care" since we want to make sure people can get it. If is available, but only to a select few who can afford it, then that is not desirable.
	35  Available Education
		% this is already covered in the Education for Everyone pattern. This seems to be more about how we need to make knowledge more easily available, which is a different pattern. It should be moved and renamed.
	36  Available Housing
	37  Available Water
?	xx  Employment   % new pattern
		% I need to address the issue of jobs or employment. It would seem important that people have something to do. Fuller talked about giving people an income -- paying them to stay home -- and that for every industrious person who produced something useful, the advantage created would pay for 100 people who didn't work. It seems naive to think that most of us would be happy not to work. I don't really think Fuller was naive, but I'm not really clear on the details of his thinking on this.
		% In Rational Optimist, the author discusses specialized jobs being the engine behind our success so far. It is distinct from being self-sufficient; the more self-sufficient you are, the poorer you are. This is different than Fuller thought about specialists, I think, though maybe not. Fuller talked about having energy slaves working for us, and this is not that far removed from that.

?	xx  Aspects of Religion   % new pattern
		% For many people, religion plays many roles in their lives. This pattern simply recognizes that religion has a place in our world but that we need to deconstruct it a bit to get to the individual contributions it makes. It helps create a sense of community, supplies ritual and a mechanism for being forgiven, helps define our ethics, provides support to the poor, and is a center of power. Some of these roles work toward a world that works and other work against it, so we need to look at each separately.

(ethics)

	38  Science of Ethics
	39  The Ethical Rule of Law
	34  Basic Human Rights
	43  Minimal Interference
?	xx  Minimal Suffering   % new pattern
	44  Universal Advantage

(belief-systems) work to ensure that the belief systems people are basing their actions on reflect reality;

	45  Rational Spiritual Life
		% maybe this should be in the middle section?
?	xx  Religious Debate   % new pattern
		% This talks about the need for rational debate about religious ideas.
?	xx  Freedom to Question Beliefs   % new pattern
		% For us to fully explore the mysteries of Universe -- Fundamental Mystery (), we need to be free to look at our spiritual experiences from a fresh perspective.
?	xx  Spiritual Community   % new pattern
		% For some of us, the only thing keeping us tied to an organized religion is the community. Whether it is a church, synagogue, mosque or other community setting, we need a community of fellow humans who are trying to come to terms with the same mysteries we are. It is  fundamental to our balance and health.

(children) encourage the idealism of youth to have the courage to embrace the kinds of change that are needed;

	46  Advantaged New Life
	47  Models of the Invisible

(design-science)

	48  Design Science Revolution
		% I'm not sure it makes sense to have this here.
	49  Comprehensive Coordination

(intellectual-resources)
% I'm thinking of these as a set of information sources that should be made available and free to all people. It is our right to have access to these records of Universe.

?	xx  Open Tools
		% the whole OT3 concept. How we need to make our current technology as open and accessible as possible.
	50  Universe
	51  Intellectual Heritage
	52  Living Library
*	xx  Inventory of Life   % new pattern
		% This exists in the form of the Encyclopedia of Life, and relates to the idea of the amateur scientist.
?	xx  The Story of Progress   % new pattern
		% The idea here is to write educational materials with a respect for the effort that came before. Also, to use some of the practices of the scientific community -- methods, cantradictions, possible issues, etc.

(tracking-success) keep an eye on what is happening and make adjustments as necessary;

?	xx  World Success Indicators   % new pattern
		% These would be similar to economic indicators but would include things like poverty rates and the like. Very much like what the UN is doing in relation to it's Millenium Development Goals.

=================================================
 PART 2
=================================================

This completes the broad patterns that define a world that works for all humanity. We now start the part of the language which gives shape to the primary tool used to create such a world — your mind. These are patterns of thinking which can help you understand the world more comprehensively and discover the unique contributions that you have to offer. These patterns are under your complete control as you are the only one who can change how you think.

% Thinking That Works
% How individuals should be


(being-an-individual) it takes a true individual to make a significant contribution to a better world;

	53  Comprehensivist
	54  Your Own Thinking
	55  Accurate Personal Universe
	56  Trained Mind
	57  Self-Directed Learner
	58  Self Discipline
?	xx  Scientific Behavior   % new pattern
		% You don't have to be a scientist to approach the world in a scientific way. In the same way, a scientist can behave in ways that are not scientific. The scientists that work to defend the idea of Creationism are examples of this.

(getting-started) to get started on your adventure, try setting up your life in a way that accommodates your new activities;

	59  Time to Think
	60  Interruptable Line of Thought
	61  Eliminating Distractions
	62  A Place to Think
	63  Home Laboratory
	64  Extended Library

(understanding-yourself) explore your mind and your unique knowledge and experiences so you can better understand yourself and how you might be able to contribute;

	65  Multiple Intelligences
	66  Language and Culture
	67  Learning Style
	68  Conscious Philosophy
	69  Tested Assumptions
	70  Emotional Literacy
	71  Perspective
	72  Audible Intuition

(passion) look for the reasons you want to work on something like this, and nurture your passions to keep you moving forward;

	73  Service to Humanity
	74  Regenerative Thought
	75  Pollyannaesque Viewpoint
	76  Support From Others
	77  Achievement Role Models
	78  Behavioral Role Model
	79  Imagination
	80  Curiosity
	81  Balanced Life
	82  Flow

(spirituality) as human beings, we are not privy to all the workings of Universe. That sense of there being more to things than what we can see is at the heart of spirituality;

	83  Fundamental Mystery

(learning) The process of learning is the process of solving problems; in both cases we start with a question and seek and answer. These patterns look at learning in the context of creating change;

	84  Learning as Problem Solving
	85  Educational Experience
	86  Learning by Understanding
	87  Rethinking
	88  Time to Absorb New Knowledge
	89  Teaching to Learn
	90  Writing to Learn
	91  Temporary Specialist

(knowledge) work to look at your knowledge in ways that help keep you flexible and open to new ideas;

	92  Knowledge as Hypothesis
	93  Accepted Ignorance
	94  Levels of Knowing
	95  High Quality Knowledge

(integrating-knowledge) as you gain more knowledge, strengthen it and extract the most meaning you can by integrating it with the other things you know;

	96  Start with Universe
		% this is really a process pattern
	97  Everything Connected
	98  Spiral of Thinking
	99  Tuned Filters
	100  Filling in the Gaps
	101  Reconciled Contradiction

(cleaning-your-universe) to create a more accurate personal universe, develop habits that can serve to find inconsistencies, false information, and avoid other cognitive biases;

	102  Active Mind
	103  Just Enough Learning
	104  Multiple Conclusions
	105  Accurate Language
	106  Susceptible to Progress
	107  Healthy Skepticism
	
*(interacting-with-others)
% I think there is room for some ideas about how we can learn from one another.

*	xx  Respectful Listener   % new pattern
*	xx  Forgiveness   % new pattern
		% I'm not sure if this a whole pattern or what. Forgiveness helps you direct your energy to more productive channels.

(landmarks) build in your mind the landmarks that will help you use your mind more effectively;

	108  Toolbox of Symbols
	109  Reference Numbers
	110  Memory Frameworks

(mind-extensions) including those tools that help us get a handle on complex ideas where there are many elements that need to be worked through;

	115  Convenient Notebook
	116  Slip File
	117  Pattern Language
	118  Extensive Cross Referencing
	119  Written Index
	120  ChronoFile
	121  Inventory

(toolkit) keep these tools close by so you can be more effective;
% I moved these up from the process section because they are really more a set of references you might want to have at your disposal. If I want them to be process-oriented, then I need to make them that way. For instance, maybe "Generalized Principles" becomes "Inventory of Generalized Principles" in the first section, and we add another pattern called 

*	137  Generalized Principles
*	138  Whole Systems
*	139  General Systems Theory


=================================================
 PART 3
=================================================

We now start the part of the language which explores how we might go about creating a world that works. Specifically, these patterns define the process of comprehensive thinking.

% This is about the process of creating change -- comprehensive anticipatory design science. As such it needs to get into the nuts and bolts of how to identify a problem, analyze it, design and implement a solution, and communicate that with the world.

% A Process That Works
% How we should be approaching the problems

(strategies) look closely at the strategies you want to use to create change in the world;

*	127  Comprehensive Thinking
*	128  Taking the Design Initiative
*	129  Work with Evolution
*	130  Trimtab
*	132  Better Tools
*	133  Reform the Environment


(inspiration) to create a change in the world, you first have to be convinced that it's something you can do;
% I'm not sure this group belongs in the process section. Maybe it does because it s integral to starting the process rather than just thinking about it. It is motivation -- what gets you moving.

	122  Sense of a Better World
	123  Power to Affect Change
	124  Vision of the Future
	125  Your Unique Contribution
	126  Response-able Mind
		% I'm not sure this belongs here.

(identifying-need) the more you begin to see the world through a comprehensive perspective, begin to look for places that can be improved;
% I moved this up because you need something to look at before you can strategize how you will do it.

*	140  Something That Needs Doing
*	141  Best Practices
*	xx  Prepare for the Black Swan   % new pattern
		% a black swan is a low probability (Fuller called them low frequency) event that cannot be predicted. Some of these, like being hit by a asteroid are important. Talk of the Nation, 21 May 2007
		% preparing for a 100 year flood or huricane is another example

(approaches) approach the problem you want to solve using different problem-solving strategies;

*	xx  A Better Mousetrap   % new pattern
		% this strategy is to look at current solutions and try to improve them in some way (new materials, higher efficiency, etc).
* 	xx  A Better Rodent-Control Industry   % new pattern
		% This strategy takes the better mousetrap to a new level where you look at the whole problem of rodent control and try to resolve systemic problems and come up with better solutions. This strategy should always be used in order to put your better mousetrap into context.
*	xx  Mouse-proof House   % new pattern
		% This approach is to go back to the original problem and see if there is a better way to frame the problem or a new family of solutions that can be explored. The objective is to make some of the old solutions completely obsolete (no mice in the hosue, no need for a mousetrap)
*	136  Ephemeralization
		% There is another pattern in section one: Growth By Ephemeralization (). I think it is different enough in scope that this one should stay. I'm thinking of this as an idea that helps complete A Better Mousetrap () in that one way to improve an existing design is to make it more efficient using better materials or techniques.

(research) learn everything you can about the problem;

*	131  One on One Contact
		% I've always thought of this as an alternative to CADS, but it is part of the process. You can help individual people all you want, but use the knowledge you gain to create something that everyone can use.

(setting-goals) once you've identified something that needs doing, decide what aspects of the problem you can explore effectively and set some goals;

	142  Spiral of Progress
	143  Anticipated Change
	144  Sculpted Scope

(planning) once you have an idea what you want to do, you will need to plan for achieving it;

	145  Plan for Success
	146  Critical Path
	147  First Things First
?	xx  The Next Design   % new pattern
		% I'm not sure what this was except that it might be related to the spiral of progress and putting an emphasis on making an improvement instead of expecting perfection.

(thinking-tools) to aid in your thinking, take advantage of some of the tools and techniques we have invented to compensate for some of the limitations of the human mind;
% I think these belong in the process section. I can see how they might be used for learning and cleaning up our Universe, but I think they represent extensions of the idea of Learning as Problem Solving; they make more sense in the context of a problem than they do just by themselves. They should probably be merged with the getting-unstuck group.

*	111  Brainstorm
*	112  Sketch
*	113  Computer Model
		% this could be a way to prototype (eh) or a way of anticipating need, but I don't think it is a way to get unstuck.
*	114  Mind Map

(getting-unstuck) use those times when you seem stuck and don't know what to do next to step back and get a bigger picture;

	148  Action Items
	149  Brain Dump
	150  Study Plan
		% maybe this should be in the learning section
	151  Thinking Out Loud
		% this might make more sense in the learning section as well, or integration

(reduce-to-practice) build your idea in the real world so you can find out if it actually works;

	152  Prototyped Capabilities

(communicating) communicate your discoveries to others and add to the intellectual heritage of humanity;

	153  Communicated Experience

(assessment) take a look at whether your design was successful;

*	134  Sufficiently Designed Technology
		% I may want to remove this pattern -- I think the problem-solving approaches replace it...
*	135  Spontaneously Adopted Design
